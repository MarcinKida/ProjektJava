\chapter{Założenia projektu}
\label{chap:Założenia projektu}

Celem projektu jest stworzenie aplikacji, która umożliwia efektywne oraz komfortowe zarządzanie magazynem, produktami, użytkownikami oraz płatnościami. Ponadto zapewnia bardzo łatwy dostęp do informacji zarówno dla administratora, jak i klienta.

Problem, jaki rozwiązałby ten program, to przede wszystkim używanie nieefektywnych metod do zarządzania magazynem oraz brak łatwego dostępu do bieżących informacji dla klientów i administratorów.

Nieefektywne metody zarządzania magazynem takie jak ręczne zliczanie sztuk, rezerwowanie miejsca czy ręczne prowadzenie dokumentacji może powodować błędy w zarządzaniu produktami, płatnościami oraz miejscem w magazynie, co skutkuje tym, że zarówno klienci jak i administratorzy byliby stanowczo niezadowoleni.

Dogłębna analiza problemu pozwala zrozumieć i naprawić problemy napotykane przez klientów oraz administratorów oraz jakie funkcje byłyby najważniejsze. Powstanie takiego programu umożliwi administratorom wygodne monitorowanie stanu magazynu oraz zaległych płatności, a użytkownikom szybki i wygodny dostęp do informacji.

\section{Wymagania funkcjonalne}
\label{sec:Wymagania funkcjonalne}

\subsection{Poziom autoryzacji}
\label{subsec:Poziom autoryzacji}
System obsługuje możliwość zarówno utworzenia nowego konta, jak i zalogowania się na już istniejące.

\subsection{Rezerwacja miejsca, sprawdzanie cennika, zmienianie danych użytkownika}
\label{subsec:Rezerwacja miejsca, sprawdzanie cennika, zmienianie danych użytkownika}
Możliwość zarezerwowania miejsca w magazynie na swój produkt, możliwość zobaczenia cen. Ponadto użytkownik może zmienić swój numer telefonu, login lub hasło.

\subsection{Funkcje administracyjne}
\label{subsec:Funkcje administracyjne}
Panel Administratora pozwala na przyznanie praw administratora dla zwykłego użytkownika, możliwość usunięcia zaległych kar oraz możliwość usunięcia z bazy danych produktów, które zarezerwował użytkownik.

\subsection{Logi serwera wszystkich istniejących danych}
\label{subsec:Logi serwera wszystkich istniejących danych}
System posiada listy użytkowników, transakcji (miesięczne przychody), kar oraz produktów w bazie danych.

\section{Wymagania niefunkcjonalne}
\label{sec:Wymagania niefunkcjonalne}

\subsection{Pojemność}
\label{subsec:Pojemność}
Pojemność jest zależna od bazy danych.

\subsection{Wydajność}
\label{subsec:Wydajność}
Interfejs użytkownika musi być w pełni przystosowany do pracy na komputerach stacjonarnych. System powinien szybko reagować na działania użytkownika, z niskimi czasami ładowania stron i minimalnymi opóźnieniami.

\subsection{Środowisko}
\label{subsec:Środowisko}
Aplikacja na pewno będzie działać na systemie Windows 11 oraz będzie kompatybilna z Java 23.

\subsection{Użyteczność}
\label{subsec:Użyteczność}
Interfejs powinien być czytelny zarówno dla administratora, jak i użytkownika.

\subsection{Skalowalność}
\label{subsec:Skalowalność}
Interfejs jest dostosowany do użytkowników korzystających z komputera stacjonarnego, bądź laptopa.

\subsection{Niezawodność}
\label{subsec:Niezawodność}
Aplikacja obsługuje wszelakie błędy i wyświetla je jako komunikat.