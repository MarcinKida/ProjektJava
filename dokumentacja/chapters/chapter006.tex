\chapter{Implementacja i przykładowe fragmenty kodu}
Oto przykładowa klasa 'WygladPrzyciskow.java'. Reszta klas znajduje się na repozytorium.


\begin{minipage}{\linewidth}
Klasa `WygladPrzyciskow` jest przykładem dobrej praktyki programistycznej, pozwalającej na tworzenie spójnych stylistycznie przycisków w interfejsie użytkownika. Dzięki niej unika się powielania kodu odpowiedzialnego za ustawianie czcionki, kolorów, ramek i innych właściwości wizualnych dla każdego przycisku z osobna. Znacząco przyspieszyło to budowanie interfejsu graficznego, co widać na przykładzie przedstawionym w \listingname~\ref{lst:WygladPrzyciskow}.

\lstinputlisting[style=javaStyle, language=Java, caption=Metoda WygladPrzyciskow, label=lst:WygladPrzyciskow]{src/styl/WygladPrzyciskow.java}
\end{minipage}

\clearpage

