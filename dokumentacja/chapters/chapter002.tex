\chapter{Założenia projektu}
\label{chap:Założenia projektu}
Celem projektu było stworzenie aplikacji desktopowej, która kompleksowo symuluje działanie magazynu.
Wizja zakładała budowę narzędzia, które z jednej strony zapewnia klientom prostą i wygodną scieżkę rezerwacji miejsca w magazynie, a z drugiej daje administratorowi rozbudowane centrum dowodzenia do zarządzania klientami oraz produktami. 

\section{Podstawowe możliwości aplikacji}
\label{sec:Wymagania funkcjonalne}
Aby zrealizowac postawione cele, aplikacja musi oferować szereg przemyślanych funkcji, które razem tworzą spójne doswiadczenie dla użytkowników. Kluczowym elementem jest proces logowania, ponieważ to na jego podstawie program wie, czy użytkownik jest administratorem, czy klientem. Każda osoba zalogowana jako klient będzie mogła zarezerwować miejsce w magazynie, sprawdzić swoje produkty bądź cennik, lecz także zmienić swoje dane bądź sprawdzić zaległe zapłaty. Z kolei każda osoba zalogowana jako Administrator będzie mogła sprawdzić listę użytkowników, nadawać uprawnienia administratora, zobaczyć przychody magazynu oraz wyświetlić listę osób, które mają zaległe kary. 

\section{Jakość i doświadczenie użytkownika}
\label{sec:Wymagania niefunkcjonalne}
Poza samymi funkcjami, fundamentalne znaczenie ma to, jak aplikacja będzie działać. Założeniem jest, aby korzystanie z niej było efektywne i bezproblemowe. Priorytetem jest, aby aplikacja była prosta w obsłudze i responsywna. Zarówno klient, jak i administrator powinni czuć,  że program działa płynnie i intuicyjnie. Kazda akcja musi wywoływać natychmiastową, przewidywalną reakcję systemu, bez irytujących opóźnień. Równie ważna jest stabilność i niezawodność. Aplikacja musi być przygotowana na nieprzewidziane sytuacje, takie jak chwilowe problemy z dostępem do bazy danych. Na koniec, dzięki zastosowaniu technologii Java, aplikacja będzie uniwersalna.

